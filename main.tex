\documentclass{report}
\usepackage[a4paper,margin=1.5in]{geometry}
\usepackage{fancyhdr}
\usepackage[titles]{tocloft}
\usepackage[titletoc]{appendix}
\usepackage{tikz}
\usepackage{xcolor}

\usepackage[ddmmyyyy]{datetime}

\usepackage{multicol}
\usepackage{amsmath}
\usepackage{amssymb}
\usepackage{amsthm}
\usepackage{pdfpages}
\usepackage{bm}
\usepackage{tikz-cd}
\usepackage{physics}
\usepackage{placeins}

%hyperref should be last apparently
\usepackage{hyperref}

\renewcommand\cftsecdotsep{\cftdot}
\renewcommand\cftsubsecdotsep{\cftdot}
\renewcommand\epsilon{\varepsilon}

% Starts a new paragraph without indentation
% and with an empty line between paragraphs
\newcommand*{\newpar}{\par\vspace{\baselineskip}\noindent}

\newcommand{\ttt}[1]{\texttt{#1}}
\newcommand{\tbf}[1]{\textbf{#1}}
\newcommand{\ul}[1]{\underline{#1}}
\newcommand{\ol}[1]{\overline{#1}}
\newcommand{\theoremname}[1]{\emph{\tbf{\ul{#1}}}}

\newcommand{\bC}{\mathbb{C}}
\newcommand{\bF}{\mathbb{F}}
\newcommand{\bN}{\mathbb{N}}
\newcommand{\bQ}{\mathbb{Q}}
\newcommand{\bR}{\mathbb{R}}
\newcommand{\bZ}{\mathbb{Z}}

\newcommand{\an}{(a_n)_{n \in \bN}}
\newcommand{\bn}{(b_n)_{n \in \bN}}
\newcommand{\sn}{(s_n)_{n \in \bN}}

\renewcommand*\contentsname{Inhalt}
\renewcommand*\proofname{Beweis}

\renewcommand{\labelenumi}{\roman{enumi}.)}

\pagestyle{fancy} %allows headers

\lhead{Emma Bach}
\rhead{\today}

\NewDocumentEnvironment{nalign}{}{\equation\aligned}{\endaligned\endequation}

\begin{document}
	% \newtheorem{codename}{printedname}[countedwith]
	\newtheorem{lemma}{Lemma}[chapter]
	\newtheorem{theorem}[lemma]{Satz}
	\newtheorem{proposition}[lemma]{Proposition}
	\newtheorem{corollary}[lemma]{Korollar}
	\newtheorem{application}[lemma]{Anwendung}
	
	\theoremstyle{definition}
	\newtheorem{definition}[lemma]{Definition}
	\newtheorem{example}[lemma]{Beispiel}

	\begin{titlepage}
	\centering
	{\Large \textsc{Vortragsskript}\par}
	\vspace{0.5cm}
	{\huge\bfseries $p$-adische Zahlen und das Henselsche Lemma\par}
    \vspace{0.5cm}
	{\Large\itshape Emma Bach\par}
	\vfill
	
	Proseminar Elementare Zahlentheorie\par
	Wintersemester 2025\par
	
	\vfill

% Bottom of the page
	{\large \today\par}
\end{titlepage}

	\tableofcontents
	\thispagestyle{fancy}
	\chapter{Umgang mit Basen ungleich $10$}
	Die gewohnte Dezimaldarstellung natürlicher Zahlen basiert auf der Erkenntnis, dass sich jede natürliche Zahl $n$ als Summe
	\begin{align*}
		n = \sum_{i = 1}^j a_i \cdot (10)^i
	\end{align*}
	von Zehnerpotenzen schreiben lässt, wobei $a_i \in \{0,1,\hdots,9\}$.
	Schreiben wir die Zahl "152", so meinen wir formell die Zahl "$1 \cdot 10^2 + 5 \cdot 10 + 2$
	Analog existiert jedoch eine ähnliche Darstellung für jedes $b \in \bN$:
	\begin{align*}
		n = \sum_{i = 1}^j a_i b^i
	\end{align*}
	mit $a_i \in \{0, 1, \hdots, b-1\}$.
	\newpar
	So ist zum Beispiel $152 = 128 + 16 + 8 = 1 \cdot 2^7 + 0 \cdot 2^6 + 0 \cdot 2^5 + 1 \cdot 2^4 + 1 \cdot 2^3 + 0 \cdot 2^2 + 0 \cdot 2 + 0$. Somit ist die Binärdarstellung von $152$ $10011000$.
	\newpar
	Es ist außerdem $152 = 2 \cdot 64 + 1 \cdot 16 + 2 \cdot 4 + 0$, also ist $[152]_{10} = [2120]_4$.
	\newpar
	Zählen in Basen ungleich $10$ funktioniert ähnlich wie in Basis $10$, nur dass man eben eine Ziffer nach links übertragen muss, wenn man $b$ erreicht (statt 10). Die Natürlichen Zahlen in Basis $3$ sind somit $0,1,2,10,11,12,20,21,22,100,\hdots$
	\chapter{Alternative Topologien auf $\bZ$ und ${\bQ}$}
	Wir betrachten die Folge:
	\begin{align*}
		a_n = 10^n
	\end{align*}
	In den reellen Zahlen mit der Standardtopologie hat diese Folge keinen Grenzwert. Betrachten wir allerdings die Glieder in Dezimaldarstellung, fällt aber intuitiv trotzdem eine Art "Grenzwertverhalten" auf:
	\begin{align*}
		a_1 = 1&0\\
		a_2 = 10&0\\
		a_3 = 100&0\\
		a_4 = 1000&0\\
		a_5 = 10000&0\\
		a_6 = 100000&0\\
		\hdots\\
		\implies \lim_{a \to \infty} a_n= ...00&0?\\
	\end{align*}
	Ein analoges "Konvergenzverhalten" sehen wir für eine beliebige Folge der Form $a_n = p^n$, solange wir die Zahlen in Basis $p$ schreiben.
	\newpar
	Unser erstes Ziel in diesem Vortrag, wird es sein, diese Form von Konvergenz und die daraus entstehenden "Zahlen mit unendlich vielen Stellen vor dem Komma" zu formalisieren. Wir werden eine Familie von Metriken $\abs{-}_p$ einführen, sodass in jeder Metrik $\abs{-}_p$ die Nullfolgen genau die Folgen sind, in denen die Basis-$p$-Darstellung immer späterer Folgenglieder in immer mehr Nullen endet.
	\newpar
	Genau so, wie in den reellen Zahlen in der Folge $0.1^n$ die $1$ "verschwindet", indem sie unendlich weit "nach rechts wandert", und die Folge somit gegen $0$ konvergiert, kann dann in unserem neuen Zahlensystem die $1$ "verschwinden", indem sie "unendlich weit nach links wandert".
	\clearpage
	\section{Der $p$-adische Betrag}
	\begin{definition}
		Sei $K$ ein Körper. Seien $x,y \in K$. Ein \tbf{Betrag} auf $K$ ist eine Funktion $\abs{-} : K \to \bR$ mit folgenden Eigenschaften:
		\begin{enumerate}
			\item $\abs{0_K} = 0$
			\item \ul{Positivität:} $x \neq 0_K \implies \abs{x} > 0$
			\item $\abs{x \cdot_K y} = \abs{x} \cdot \abs{y}$
			\item \ul{Subadditivität:} $\abs{x +_K y} \leq \abs{x} + \abs{y}$
		\end{enumerate}
	\end{definition}
	\begin{proposition}
		Sei $K$ ein Körper und $\abs{}$ ein Betrag auf $K$. So ist
		\begin{align*}
			d : K \times K &\to \bR\\
			(x,y) &\mapsto \abs{x - y}\\
		\end{align*}
		eine Metrik auf $K$.
	\end{proposition}
	\iffalse
	\begin{definition}
		Sei $K$ ein Körper und  $G$ eine linear geordnete Gruppe. Eine \tbf{Bewertung auf $K$} ist eine Abbildung $v : K \to G \cup \{\infty\}$ mit folgenden Eigenschaften:
		\begin{enumerate}
			\item $v(0) = \infty$
			\item $g \neq 0 \implies v(g) \neq \infty$ 
			\item $v(a \cdot_K b) = v(a) +_G v(b)$ (Bewertungen sind Gruppenhomomorphismen aus der Multiplikativen Gruppe von $K$ in $G$)
			\item $v(a +_K b) \geq \min(v(a), v(b))$ 
			\item $v(a) = v(b) \implies v(a +_K b) > \min(v(a), v(b))$
			\item $v(a) \neq v(b) \implies v(a +_K b) = \min(v(a), v(b))$
		\end{enumerate}
	\end{definition}
	\fi
	\begin{definition}
		Sei $p$ eine Primzahl. Die \tbf{$p$-adische Bewertung} auf $\bZ$ ist folgende Abbildung:
		\begin{align*}
			v_p(n) = 
			\begin{cases}
				\max\{k \in \bN_0 : p^k \mid n\} & n \neq 0\\
				\infty & n = 0
			\end{cases}
		\end{align*}
		Die $p$-adische Bewertung ist auch bekannt als die \tbf{Vielfachheit von $p$ in $n$}.
	\end{definition}
	\begin{example}
		Es gilt:
		\begin{itemize}
			\item $v_5(25) = v_5(5^2) = 2$
			\item $v_5(50) = v_5(2 \cdot 5^2) = 2$.
			\item $v_5(25 + 50) = v_5(75) = v_5(3 \cdot 5^2) = 2$
			\item $v_5(50 + 75) = v_5(125) = v_5(5^3) = 3$.
		\end{itemize} 
	\end{example}
	Intuitiv gibt uns die $p$-adische Bewertung einer ganzen Zahl $n$ also die größte Potenz von $p$, durch die $n$ teilbar ist.
	\newpar
	Die $p$-adische Bewertung ist das Standardbeispiel einer Bewertung. Es existieren viele andere wichtige Bewertungen in der kommutativen Algebra, in der komplexen Analysis und in der algebraischen Geometrie, diese sind allerdings in der Regel leider nichttrivial und würden den Rahmen dieses Proseminarvortrags sprengen.\\
	\begin{definition}
		Wir erweitern die $p$-adische Bewertung auf ganz $\bQ$ durch:
		\begin{align*}
			v_p\left(\frac{r}{s}\right) = v_p(r) - v_p(s)
		\end{align*}
	\end{definition}
	\noindent Dies entspricht der Potenz von $p$, die wir erhalten, wenn wir alle Potenzen von $p$ aus dem Bruch "rausziehen".
	\begin{example}
		\begin{align*}
			v_5\left(\frac{75}{125}\right) &= 2 - 3 = -1
		\end{align*}
		bzw.
		\begin{align*}
			v_5\left(\frac{75}{125}\right) &= v_5\left(\frac{3}{5}\right)\\ &= v_5\left(5^{-1} \cdot \frac{3}{1}\right)\\
			&= -1
		\end{align*}
	\end{example}
	\begin{definition}
		Wir definieren den $p$-adischen Betrag $\abs{-}_p$ auf $\bQ$ durch:
		\begin{align*} 
			\abs{n}_p = \frac{1}{p^{v_p(n)}}
		\end{align*}
	\end{definition}
	\noindent Der $p$-adische Betrag einer rationalen Zahl ist also genau dann gering, wenn ihre Darstellung in Basis $p$ in vielen aufeinanderfolgenden Nullen endet.
	\newpar
	Wir notieren die dazugehörige Metrik als $d_p(x,y)$. In dieser Metrik sind zwei Zahlen genau dann nah aneinander, wenn für ein Großes $n$ die letzten $n$ Ziffern ihrer Darstellung in Basis $p$ identisch sind.
	\begin{example}
		\begin{align*}
			d_2(24, 25) &= d_2([11.000]_2,[11.001]_2)\\
			 			&= \abs{1}_2\\ 
			 			&= \frac{1}{2^0}\\ 
			 			&= 1\\\\
			d_2(24, 1816) &= \abs{1792}_2\\ 
						  &= \abs{7 \cdot 2^8}\\
						  &= \frac{1}{2^8} = \frac{1}{256}\\\\
			d_2(24, 1816) &= d_2([11.000]_2, [11100011000]_2)\\
						  &= \Big|[-11.100.000.000]_2\Big|_2\\
						  &= \frac{1}{2^8} = \frac{1}{256}\\
		\end{align*}
	\end{example}
	\begin{theorem}
		Der $p$-adische Betrag erfüllt die \tbf{ultrametrische Dreiecksungleichung}:
		\begin{align*}
			\abs{x + y}_p \leq \max(\abs{x}_p,  \abs{y}_p)
		\end{align*}
		Wir nennen einen solchen Betrag \tbf{nichtarchimedisch}.
	\end{theorem}
	\begin{example}
		\begin{align*}
			\abs{54}_3 &= \abs{2 \cdot 2^3} = \frac{1}{27}\\
			\abs{81}_3 &= \abs{3^5}_3 = \frac{1}{81}\\
			\abs{54 + 81}_3 &= \abs{135}_3 = \abs{5 \cdot 3^3}_3 = \frac{1}{27}
		\end{align*}
	\end{example}
	\begin{proposition}
		\theoremname{Satz von Ostrowski:} Jeder Betrag auf $\bQ$ ist entweder:
		\begin{itemize}
			\item Der triviale Betrag $\abs{x}_0 = 
			\begin{cases}
				0 & x = 0_K\\
				1 & x \neq 0_K\\
			\end{cases}$, 
			\item oder äquivalent zu $\abs{-}_p$ für eine Primzahl $p$, 
			\item oder äquivalent zum Standardabsolutbetrag $\abs{-}$.
		\end{itemize} 
	\end{proposition}
	\chapter{Die $p$-adischen Zahlen}
	\section{Konstruktion der $p$-adischen Zahlen}
	\begin{definition}
		Wir bezeichnen die Vervollständigung des Rings $\bZ$ gemäß der $p$-adischen Metrik als die \tbf{$p$-adischen} ganzen Zahlen $\bZ_p$. Analog bezeichnen wir die Vervollständigung des Rings $\bQ$ gemäß der $p$-adischen Metrik als die \tbf{$p$-adischen} Zahlen $\bQ_p$
	\end{definition}
	\noindent Wir erhalten die gefragten Vervollständigungen durch eine Äquivalenzklassenkonstruktion von Cauchyfolgen, in der Cauchyfolgen äquivalent sind, wenn ihre Differenz eine Nullfolge ist, analog zur Konstruktion von $\bR$ als Vervollständigung gemäß des Standardabsolutbetrags.
	\begin{proposition}
		$\bZ_p$ ist ein Ring, der $\bZ$ als dichte Teilmenge enthält. $\bQ_p$ ist ein Körper, der $\bQ$ als dichte Teilmenge enthält.
	\end{proposition}
	\begin{proposition}
		\begin{align*}
			\bZ_p = \{z \in \bQ_p : \abs{z}_p \leq 1\} = \{z \in \bQ_p : \nu_p(z) \geq 0\}
		\end{align*}
	\end{proposition}
	\begin{proposition}
		Jede Reihe der Form
		\begin{align*}
			x = \sum_{n = m}^\infty d_n p^n,
		\end{align*}
		wobei $m \in \bZ_{\leq 0}$, $d_n \in \bZ / p\bZ$, konvergiert in $\bQ_p$. Wir nennen die Folge $d_n$ die \tbf{$p$-adische Darstellung von $x$}.
	\end{proposition}
	\noindent Wir schreiben eine $p$-adische Zahl $z$ analog zur Standarddarstellung Basis $p$ als 
	\begin{align*}
		z = \hdots d_4 d_3 d_2 d_1 d_0, d_{-1} \hdots d_{m}
	\end{align*}
	mit der kleinsten Ziffer rechts. In manchen Quellen werden $p$-adische Zahlen umgekehrt geschrieben, mit der kleinsten Ziffer links.
	\newpar
	\noindent Die $p$-adischen ganzen Zahlen sind genau die $p$-adischen Zahlen mit $m = 0$. Die Zahlen $\bZ \subsetneq \bZ_p$ sind genau die Zahlen, deren $p$-adische Darstellungen endlich sind, und diese Darstellung ist genau die Übliche.
	\begin{example}
		Wir wollen die $2$-adische Darstellung von $-1$ finden.
		\newpar
		Wir brauchen also eine Folge $d_n$, sodass für jedes $n$
		\begin{align*}
			d_0 + d_1 2 + \hdots + d_n 2^n \equiv -1 \mod 2^{n+1}
		\end{align*}
		Für $n = 0$ erhalten wir:
		\begin{align*}
			-d_0 \equiv -1 \mod 2
		\end{align*}
		also $d_0 = 1$. Wir wollen induktiv Zeigen, dass alle anderen Ziffern ebenfalls $1$ sind. Angenommen, $d_0 = d_1 = \hdots = d_{n-1} = 1$. Dann gilt:
		\begin{align*}
			d_0 + 2d_1 + \hdots + 2^nd_n  = \left(\sum_{i = 0}^{n-1} 2^i + d_n 2^n\right) 
			&\equiv -1 \mod 2^{n+1}\\
			\implies \qquad (2^n - 1) - d_n 2^n &\equiv -1 \mod 2^{n+1}\\
			\implies \qquad 2^n - d_n 2^n  &\equiv 0 \mod 2^{n+1}\\
			\implies \qquad 1 - d_n &\equiv 0 \mod 2^{n+1}
		\end{align*}
		Also $d_n = 1$. Die $2$-adische Darstellung von $-1$ ist also $\hdots1111$.
	\end{example}
	\begin{proposition}
		Als Hausaufgabe: In der $p$-adische Darstellung von $-1$ für beliebiges $p$ ist jede Ziffer $p-1$. Die $5$-adische Darstellung ist also $\hdots4444$ und die $7$-adische Darstellung ist $\hdots66666$.
	\end{proposition}
	\noindent Diese Darstellung ist tatsächlich analog zur Darstellung negativer Zahlen in binär in den meisten modernen Computern. Negative Zahlen werden in der Regel im \tbf{Zweierkomplement} dargestellt, was bedeutet, dass das höchstmögliche Bit negativ gezählt wird. Da wir hierfür vorher ein "höchstmögliches Bit" wählen müssen, ist unser Zahlenbereich beschränkt, und die Addition großer Zahlen kann zu Problemen wie dem bekannten \textit{Integer Overflow} führen. Die $p$-adischen Zahlen lösen dieses Problem, indem sie Zahlen erlauben, die beliebig Hochwertige Ziffern enthalten.
	\chapter{Das Henselsche Lemma}
	Das Henselsche Lemma ist eine Methode, um Polynomgleichungen in $\bZ_p$ zu lösen. Es liefert uns somit insbesondere eine einfache Möglichkeit, die $p$-adische Darstellung bestimmter algebraischer Zahlen zu finden.
	\begin{theorem}
		Sei 
		\begin{align*}
			f(x) = \sum_{i = 0}^n c_i x^i
		\end{align*}
		ein Polynom mit Koeffizienten $c_i \in \bZ_p$. Sei $f'(x)$ die Ableitung von $f(x)$, also
		\begin{align*}
			f'(x) = \sum_{i = 0}^{n-1} ic_{i+1} x^i
		\end{align*}
		Sei außerdem $a \in \bZ_p$, sodass:
		\begin{align*}
			f(a) \equiv 0 &\mod p\\
			f'(a) \not\equiv 0 &\mod p
		\end{align*}
		Dann existiert ein eindeutiges $\alpha \in \bZ_p$, sodass:
		\begin{align*}
			\begin{array}{rll}
				f(\alpha) &= 0\\
				\alpha &\equiv a &\mod p
			\end{array}
		\end{align*}
	\end{theorem}
	\begin{proof}
		Wir konstruieren eine eindeutige Folge $a_n$ in $\bZ_p$, sodass:
		\begin{enumerate}
			\item $f(a_n) \equiv 0 \mod p^{n + 1}$
			\item $a_n \equiv a_{n - 1} \mod p$
			\item $a_n \in  \{0, \hdots, p^{n+1} - 1\}$
		\end{enumerate}
	\noindent Daraufhin werden wir zeigen, dass $\displaystyle \lim_{n \to \infty} a_n = \alpha$.
	\newpar
	Sei erst einmal als Induktionsbasis $a_0 \in \{0, \hdots, p - 1\}$ mit $a_0 \equiv a \mod p$.
	\newpar
	Seien nun $a_0, \hdots, a_{n-1}$ bereits mit den gewünschten Eigenschaften konstruiert.
	Aus ii.) folgt, dass $a_n$ die Form $a_{n-1} + b_np$ haben muss, und aus iii.) folgt $b_n \in \{0, \hdots, p^n-1\}$.
	\newpar 
	Also gilt:
	\begin{align*}
		f(a_n) = f(a_{n - 1} + b_np^n) &= \sum_{i = 0}^n c_i(a_{n - 1} + b_np^n)^i\\
			                         &\equiv \sum_{i = 0}^n c_i(a_{n-1}^i + i(a_{n-1}^{i-1})(b_np^n)) \mod p^{n+1} \quad\text{(Binomischer Lehrsatz)}\\
			                    	 &= \sum_{i = 0}^n c_ia_{n-1}^i + \left(\sum_{i = 0}^nic_i a_{n-1}^{i-1}\right)b_np^n\\
			                    	 &= f(a_{n-1}) + f'(a_{n-1}) b_n p^n
	\end{align*}
	(Dies ist eine Taylorapproximation erster Ordnung - wir sehen, dass diese in diesem Fall exakt ist!)
	\newpar
	Da $f(a_{n-1}) \equiv 0 \mod p^n$ per Annahme gilt $f(a_{n-1}) \equiv kp \mod p^n$ für ein $k \in \{0, \hdots, p-1\}$.
	\newpar
	Für $f(a_n) \equiv 0 \mod p^{n+1}$ brauchen wir also $kp + f'(a_0) b_n p^n \equiv 0 \mod p^{n+1}$. Klammern wir $p$ aus, sehen wir dass dies gegeben ist, falls:
	\begin{align*}
		k + f'(a_{n-1}) b_n \equiv 0 &\mod p\\
		\implies f'(a_{n-1}) b_n \equiv -k &\mod p\\
	\end{align*}
	Es gilt per Induktionsannahme $a_{n-1} \equiv a_0 \mod p$, also $f'(a_{n-1}) \equiv f'(a_0) \not\equiv 0 \mod p$, also existiert die Lösung
	\begin{align*}
		b_n \equiv -\frac{k}{f'(a_{n-1})} \mod p
	\end{align*}
	Es bleibt noch zu Zeigen, dass $\alpha = a_0 + b_1p + b_2p^2 \hdots$ eine exakte $p$-adische Lösung ist. 
	\newpar
	Es gilt $\alpha \equiv a_0 \equiv a \mod p$, und es gilt $f(\alpha) \equiv 0 \mod p^n$ für alle $n$, also $\abs{f(\alpha)}_p < \frac{1}{p^n}$ für alle $n$, also $\abs{f(\alpha)}_p = 0$. Somit ist $\alpha$ tatsächlich eine Lösung.
	\newpar
	Es bleibt noch die Eindeutigkeit der Lösung zu zeigen. Angenommen, $\beta$ sei eine weitere Lösung, also $f(\beta) = 0$ und $\beta \equiv a \mod p$. Aus der zweiten Bedingung folgt bereits $\alpha \equiv \beta \mod p$.
	\newpar
	Angenommen, $\alpha \equiv \beta \mod p^n$. Dann gilt $\beta = \alpha + p^n \gamma_n$ mit $\gamma_n \in \bZ_p$. Dieselbe Polynomerweiterung, welche bereits im bereits im Beweis verwendet wurde, liefert:
	\begin{align*}
		f(\beta) = f(\alpha + p^n \gamma_n) \equiv f(\alpha) + f'(\alpha)p^n\gamma_n \mod p^{n + 1}
	\end{align*}
	Es gilt $f(\beta) = 0$, also $f'(\alpha)\gamma_n \equiv 0 \mod p$. Wir wissen $f'(\alpha) \equiv f'(a) \not \equiv 0 \mod p$, also $\gamma_n \equiv 0 \mod p$, also $\alpha \equiv \beta \mod p^{n + 1}$.
	\newpar
	Somit gilt $\alpha = \beta$, also ist die Lösung eindeutig.
	\end{proof}
	\clearpage
	\section{Konstruktion $p$-adischer Darstellungen}
	Hensel's Lemma liefert uns einen praktischen Weg, $p$-adische Darstellungen vieler algebraischer Zahlen zu konstruieren.
	\iffalse
	\begin{application}
		Wir wollen die $2$-adische Darstellung von $-1$ bestimmen. Wir zeigen Induktiv, dass jede Ziffer $1$ ist. Wir betrachten zuerst das Polynom $f(x) = x + 1$. Es gilt $f'(x) = 1 \not\equiv 0 \mod 2$. Es gilt außerdem $f(1) = 2 \equiv 0 \mod 2$. Wähle also $a_0 = 1$.
		\newpar
		Es gilt $f(1) = 2 \overset{!}{\equiv} k \cdot 2 \mod 2^2 \implies k = 1$, also $b_1 \equiv -1 \mod 2 = 1$. Sei nun angenommen, $a_0 = \hdots = a_{n-1} = 1$. Dann gilt:
		\begin{align*}
			f(b_{n-1}2^{n-1} + \hdots + b_1 2 + a_0) = f\left(\sum_{i = 1}^{n-1} 2^i\right) = 2^n - 1 + 1 = 2^n \overset{!}{\equiv} k \cdot 2^{n} \mod 2^{n+1}
		\end{align*}
		also wieder $k = 1$ und $b_n = 1$.
		\newpar
		Die $2$-adische Darstellung von $-1$ ist also $\hdots11111$.
	\end{application}
	\begin{application}
		Bezug zum Zweierkomplement in der Informatik.
	\end{application}
	\begin{proposition}
		In der $p$-adische Darstellung von $-1$ ist jede Ziffer $p-1$. Die $5$-adische Darstellung ist also $\hdots44444$, die $7$-adische Darstellung ist $\hdots66666$, und so weiter.
	\end{proposition}
	\fi
	\begin{example}
		Wir wollen $\sqrt{2} \in \bZ_7$ finden. Sei also $f(x) = x^2 - 2$, also $f'(x) = 2x$. Wir suchen zuerst unser $a = a_0$. Da $f'(a) \not\equiv 0 \mod 7$ brauchen wir $2a \not\equiv 0$. Die Bedingung $f(a) \equiv 0 \mod 7$ liefert:
		\begin{align*}
			a^2 - 2 \equiv 0 \mod 7
		\end{align*}
		also $a^2 \equiv 2 \mod 7$. Eine Möglichkeit ist $3$, eine weitere Möglichkeit ist $4 \equiv -3 \mod 7$. Da also Nullstellen existieren, garantiert das Henselsche Lemma eine Lösung - wir haben also eine irrationale reelle Zahl gefunden, welche in den $p$-adischen ganzen Zahlen enthalten sind.
		\newpar
		Wir wollen nun die letzten paar Ziffern berechnen. Wir entscheiden uns für die positive Wurzel. Nun wollen wir $k$ bestimmen, sodass
		\begin{align*}
			f(3) = 7 \equiv 7k \mod 49
		\end{align*}
		Es reicht also $k = 1$. Nun gilt:
		\begin{align*}
			b_1 &\equiv -\frac{1}{f'(3)} \mod 7\\
			\implies b_1 &\equiv -\frac{1}{6} \mod 7\\
			\implies 6b_1 + 1 &\equiv 0 \mod 7\\
			\implies b_1 = 1\\
		\end{align*}
		Also $a_1 = 3 + 1 \cdot 7 = 10$.
		Für die dritte Ziffer brauchen wir $f(a_1) = f(10) = 98 \equiv 49k \mod 343$, also $k = 2$. Nun gilt:
		\begin{align*}
			b_2 &\equiv -\frac{2}{f'(10)} \mod 7 \\
				&\equiv -\frac{2}{20} \mod 7 \\
				&\equiv -\frac{1}{10} \mod 7 \\
			\implies 10b_2 + 1 &\equiv 0 \mod 7\\
			\implies 3b_2 + 1 &\equiv 0 \mod 7\\
			\implies b_2 &\equiv 2 \mod 7\\
		\end{align*} 
		Also $a_2 = 3 + 1 \cdot 7 + 2 \cdot 7^2 = 108$.
		\newpar
		Die $7$-adische Darstellung von $\sqrt{2}$ endet also in $\hdots213$.
	\end{example}
	\begin{application}
		Eine $p$-adische Zahl $u$ hat eine $k$-te Wurzel in den $p$-adischen Zahlen, wenn $k \not \equiv 0 \mod p$ und eine Zahl $n$ mit $n \equiv u \mod p$ existiert, sodass $n$ eine $k$-te Wurzel in $\bZ / p\bZ$ hat.
	\end{application}
	\begin{proof}
		Wähle $f(x) = x^k - u$. So gilt $f'(x) = kx \not \equiv 0$, und das Henselsche Lemma garantiert eine Lösung, falls wir eine initiale Nullstelle $n = x^k \equiv u \mod p$ finden können
	\end{proof}
	\begin{application}
		$i = \sqrt{-1} \in \bZ_3$
	\end{application}
	\begin{proof}
		Wähle $f(x) = x^2 + 1$, also $f'(x) = 2x$. Wir wollen $x^2 \equiv -1 \equiv 2 \mod 3$. Unsere Möglichkeiten sind also wieder $3$ und $4$, wir wählen wieder $a_0 = a = 3$. Es gilt $f'(x) = 6 \not\equiv 0 \mod 7$, also existiert eine eindeutige $p$-adische Erweiterung von $a$, deren Quadrat $-1$ ist.
	\end{proof}
	\iffalse
	\section{Eine verallgemeinerte Version}
	Unsere erste Version des Henselschen Lemmas erlaubt uns nicht, Wurzeln mit Vielfachheiten größer als $1$ als die initiale Approximation zu verwenden. Wir beweisen nun eine allgemeinere Version, welche uns außerdem eine Abschätzung des $p$-adischen Abstands zwischen der Approximation und der exakten Lösung gibt:
	\begin{theorem}
		Sei 
		\begin{align*}
			F(x) = \sum_{i = 0}^n c_i x^i
		\end{align*}
		ein Polynom mit Koeffizienten $c_i \in \bZ_p$. Sei $F'(x)$ die Ableitung von $F(x)$, also
		\begin{align*}
			F'(x) = \sum_{i = 0}^{n-1} c_{i+1} x^i
		\end{align*}
		Sei außerdem $a_0 \in \bZ_p$, sodass:
		\begin{align*}
			\abs{F(a_0)}_p < \abs{F'(a_0)}_p^2 
		\end{align*}
		Dann existiert ein eindeutiges $a \in \bZ_p$, sodass:
		\begin{align*}
				F(a) &= 0\\
				\abs{a - a_0}_p & = \abs{\frac{F(a_0)}{F'(a_0)}}_p < \abs{F'(a)}_p\\
				\abs{F'(a)}_p &= \abs{F'(a_0)}_p
		\end{align*}
	\end{theorem}
	\fi
	\appendix
	\chapter{Quellen}
	Hauptquellen sind "p-adic Numbers, p-adic Analysis and Zeta Functions", geschrieben von Neal Koblitz, "Algebraic Number Theory", geschrieben von Jürgen Neukirch, und Keith Conrads Notizen zum Henselschen Lemma. Eine weiter Quelle ist \href{https://math.uchicago.edu/~may/REU2020/REUPapers/Zheng,Yiduan.pdf}{$p$-adic Numbers, $Q_p$ and Hensels Lemma}, geschrieben von Yiduan Zheng.
\end{document}