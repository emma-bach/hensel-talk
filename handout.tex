\documentclass[9pt]{article}
\usepackage[a4paper,margin=1.2in]{geometry}
\usepackage{fancyhdr}
\usepackage[titles]{tocloft}
\usepackage[titletoc]{appendix}
\usepackage{tikz}
\usepackage{xcolor}

\usepackage[ddmmyyyy]{datetime}

\usepackage{multicol}
\usepackage{amsmath}
\usepackage{amssymb}
\usepackage{amsthm}
\usepackage{pdfpages}
\usepackage{bm}
\usepackage{tikz-cd}
\usepackage{physics}
\usepackage{placeins}

\usepackage[german]{babel}
\usepackage{algorithm}
\usepackage{algpseudocode}

\usepackage[many]{tcolorbox}
\tcolorboxenvironment{proof}{
	colback=black!5!white,
	boxrule=0pt,
	sharp corners,
	breakable,
	enhanced,
}

%hyperref should be last apparently
\usepackage{hyperref}

\renewcommand\cftsecdotsep{\cftdot}
\renewcommand\cftsubsecdotsep{\cftdot}
\renewcommand\epsilon{\varepsilon}

% Starts a new paragraph without indentation
% and with an empty line between paragraphs
\newcommand*{\newpar}{\par\vspace{\baselineskip}\noindent}

\newcommand{\ttt}[1]{\texttt{#1}}
\newcommand{\tbf}[1]{\textbf{#1}}
\newcommand{\ul}[1]{\underline{#1}}
\newcommand{\ol}[1]{\overline{#1}}
\newcommand{\theoremname}[1]{\emph{\tbf{\ul{#1}}}}

\newcommand{\bC}{\mathbb{C}}
\newcommand{\bF}{\mathbb{F}}
\newcommand{\bN}{\mathbb{N}}
\newcommand{\bQ}{\mathbb{Q}}
\newcommand{\bR}{\mathbb{R}}
\newcommand{\bZ}{\mathbb{Z}}

\newcommand{\an}{(a_n)_{n \in \bN}}
\newcommand{\bn}{(b_n)_{n \in \bN}}
\newcommand{\sn}{(s_n)_{n \in \bN}}

\renewcommand*\contentsname{Inhalt}
\renewcommand*\proofname{Beweis}

\renewcommand{\labelenumi}{\roman{enumi}.)}

%\pagestyle{fancy} %allows headers
\pagenumbering{gobble}

\begin{document}
	% \newtheorem{codename}{printedname}[countedwith]
	\newcounter{hahaone}
	\setcounter{hahaone}{1}
	
	\newtheorem{lemma}{Lemma}[hahaone]
	\newtheorem{theorem}[lemma]{Satz}
	\newtheorem{proposition}[lemma]{Proposition}
	\newtheorem{corollary}[lemma]{Korollar}
	\newtheorem{application}[lemma]{Anwendung}
	
	\theoremstyle{definition}
	\newtheorem{definition}[lemma]{Definition}
	\newtheorem{example}[lemma]{Beispiel}
	
	\title{\Large\vspace{-3cm}\textbf{$p$-adische Zahlen und das Henselsche Lemma}}
	\author{Emma Bach - Proseminar Elementare Zahlentheorie, WS25/26}
	
	\maketitle
	

	\begin{definition}
		Sei $p$ eine Primzahl. Wir nennen folgende Abbildung $\bZ \to \bN_0$ die \tbf{$p$-adische Bewertung} auf $\bZ$:
		\begin{align*}
			v_p(n) = 
			\begin{cases}
				\max\{k \in \bN_0 : p^k \mid n\} & n \neq 0\\
				\infty & n = 0
			\end{cases}
		\end{align*}
		Die $p$-adische Bewertung ist auch bekannt als die \tbf{Vielfachheit von $p$ in $n$}. Die $p$-adische Bewertung kann durch die Vorschrift $v_p\left(\frac{r}{s}\right) = v_p(r) - v_p(s)$ auf die rationalen Zahlen erweitert werden.
	\end{definition}
	\begin{definition}
		Der $p$-adischen Betrag $\abs{-}_p$ auf $\bQ$ ist die Abbildung:
		\begin{align*} 
			\abs{n}_p = \frac{1}{p^{v_p(n)}}
		\end{align*}
	\end{definition}
	\begin{theorem}
		\theoremname{Satz von Ostrowski:} Jeder Betrag auf $\bQ$ ist entweder der triviale Betrag, oder äquivalent zu $\abs{-}_p$ für eine Primzahl $p$, oder äquivalent zum Standardabsolutbetrag $\abs{-}$.
	\end{theorem}
	\setcounter{hahaone}{2}
	\begin{definition}
		Wir bezeichnen die Vervollständigung des Rings $\bZ$ gemäß der durch den $p$-adischen Absolutbetrag erzeugten Metrik als die \tbf{$p$-adischen ganzen Zahlen} $\bZ_p$. Analog bezeichnen wir die Vervollständigung des Rings $\bQ$ gemäß der $p$-adischen Metrik als die \tbf{$p$-adischen Zahlen} $\bQ_p$. Die Konstruktion verläuft analog zur Konstruktion von $\bR$ aus Cauchyfolgen in $\bQ$.
	\end{definition}
	\begin{proposition}
		Jede Reihe der Form
		\begin{align*}
			x = \sum_{n = m}^\infty d_n p^n,
		\end{align*}
		wobei $m \in \bZ_{\leq 0}$, $d_n \in \{0,1,\hdots,p-1\}$, konvergiert in $\bQ_p$. Wir nennen die Folge $d_n$ die \tbf{$p$-adische Darstellung von $x$}. Jede $p$-adische Zahl kann als eine solche Reihe dargestellt werden.
	\end{proposition}
	\noindent Wir können jede $p$-adische Zahl $z$ somit analog zur Standarddarstellung Basis $p$ schreiben: 
	\begin{align*}
		z = \hdots d_4 d_3 d_2 d_1 d_0, d_{-1} \hdots d_{m}
	\end{align*}
	In manchen Quellen werden $p$-adische Zahlen umgekehrt geschrieben, mit der kleinsten Ziffer links.
	\begin{proposition}
		Sei $p$ beliebig. So ist in der $p$-adischen Darstellung von $-1$ jede Ziffer\\ $p-1$. Die $5$-adische Darstellung von $-1$ ist also $\hdots4444$ und die $7$-adische Darstellung ist $\hdots66666$. Somit ist bei der Darstellung $p$-adischer Zahlen kein Vorzeichen nötig.
	\end{proposition}
	\setcounter{hahaone}{3}
	\begin{theorem}
		\theoremname{Henselsches Lemma:}
		Sei $f(x)$ ein Polynom mit Koeffizienten $c_i \in \bZ_p$. Sei $f'(x)$ die Ableitung von $f(x)$.
		Sei außerdem $a \in \bZ_p$, sodass:
		\begin{align*}
			f(a) \equiv 0 &\mod p\\
			f'(a) \not\equiv 0 &\mod p
		\end{align*}
		Dann existiert ein eindeutiges $\alpha \in \bZ_p$, sodass:
		\begin{align*}
				f(\alpha) &= 0\\
				\alpha &\equiv a \mod p
		\end{align*}
	\end{theorem}
	\begin{application}
		Das Polynom $f(x) = x^2 + 1$ erfüllt für $p = 5$ und $a = 2$ die gefragten Bedingungen. Somit existiert eine Nullstelle in $\bZ_5$, also $i = \sqrt{-1} \in \bZ_5$.
	\end{application}
	\begin{application}
		Sei $u \in \bZ_p$. Sei $k \not \equiv 0 \mod p$. Sei $n$ eine Zahl mit $n \equiv u \mod p$, welche eine $k$-te Wurzel in $\bZ / p\bZ$ hat. Dann hat $u$ eine $k$-te Wurzel in $\bZ_p$.
	\end{application}
\end{document}